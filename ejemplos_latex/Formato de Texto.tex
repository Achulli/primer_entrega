\documentclass[11pt]{article}
\begin{document}
\tableofcontents
\title{Documento \LaTeX \ de practica}
\author{Autor}
\date{\today}

\maketitle

Esto produce texto en \textit{cursiva}

Esto produce texto en \textbf{negrita}

Esto produce texto en \textsc{small caps}

Esto produce texto con tipografia \texttt{typewriter}

Please excuse my \begin{tiny}dear aunt sally
\end{tiny}

Please excuse my dear aunt sally

Please excuse my \begin{large}dear aunt sally
\end{large}

Please excuse my \begin{Large}dear aunt sally
\end{Large}

Please excuse my \begin{LARGE}dear aunt sally
\end{LARGE}

Please excuse my \begin{huge}dear aunt sally
\end{huge}

Please excuse my \begin{Huge}dear aunt sally
\end{Huge}

\begin{center}
Centrado
\end{center}

\begin{flushleft}
Justificado a la izquierda
\end{flushleft}

\begin{flushright}
Justificado a la derecha
\end{flushright}

\section{Funciones Lineales}
	\subsection{Forma pendiente - ordenada al origen}
	La forma pendiente - ordenada al origen de una funcion lineal esta dada por $y = ax + b$
	\subsection{Forma estandar}
	\subsection{Forma pendiente - punto}
\section{Funciones Cuadraticas}
	\subsection{Forma vertice}
	\subsection{Forma estandar}
	\subsection{Forma factorizada}


\end{document}